% $  Id: validation.tex  $
% !TEX root = main.tex


%%
\section{Validation}
\label{sec:validation}

To validate the appropriateness of CollabIDE, we define two research questions
\begin{enumerate*}[label=(\arabic*)]
\item improve productivity, and
\item SPL
\end{enumerate*} 
To answer these questions, we conducted an empirical evaluation 
to measure how effective CollabIDE was at reducing overhead problems 
in both Distributed Software Development and Software Product Lines.

%%%%
\subsection{Experiment Design}

\authorcomment[missing]{NC}{Context, Research questions, analysis method, data collection method}

For each research question, we designed one experiment. In each experiment, we asked 
a group of developers to solve programming tasks using either CollabIDE or a 
conventional IDE following a set of instructions that aimed to emulate the workflow that is carried out in each development model. The experiments differed in team sizes, the tasks that needed to be completed and the workflow each set of participants had to follow. Two metrics were taken in each experiment, first, the percentage of completion of the total tasks assigned and second, the amount of time in minutes spent performing actions related to version control.

%%%%
\subsubsection{Experiment 1: Software development in a distributed set-up}
The objective of this experiment was to quantify the overhead reduction that can be achieved with CollabIDE in a distributed development project where team members must use version control constantly to always have up to date code. 
For the experiment, groups of developers had to use JavaScript to accomplish their given programming tasks and must also had to use version control periodically so that both team members code remained up to date through the experiment. 


%%%%
\subsubsection{Experiment 2: Product variants development and deployment}
In this second experiment the objective was to also quantify the overhead reduction that can be achieved with CollabIDE within a product line development context where various variants of a product must be maintained.
For the experiment, individual developers had to program in Java or JavaScript depending on the IDE they used to complete their tasks. Additionally, they had to use version control to manage the different product variants that were involved in the programming tasks.

\begin{figure}[htbp]
  \centering
  \includegraphics[width=0.7\textwidth]{img/collabIDEGeneral}
  \caption{CollabIDE tool}
  \label{fig:collabide}
\end{figure}

%%%%
\subsection{Experiment Setup}

Six developers were gathered for the Distributed Software Development experiment, then they were split into groups of two. Two groups would be using CollabIDE and the remaining group would be using Sublime Text in conjunction with git and github. The programming task for this experiment was to implement a set of common graph algorithms using only JavaScript. The participants were given a total of ninety minutes to accomplish this task. Each group was required to obtain their partner’s changes every fifteen minutes using their tools at hand.
In the Software Product Lines experiment, a different group of four developers was gathered. Two of them would be using CollabIDE and the other two would be using Eclipse in conjunction with git and github. In this case, the programming task was to implement a set of data structures with some basic functionality using JavaScript (CollabIDE) or Java (Eclipse). Each data structure also had to be a variant of a given base structure. These participants were also given ninety minutes to complete their task. In order to manage the different product variants, the participants were requested to use version control in each of their IDEs.


\authorcomment[missing]{NC}{Tools, exercise}
	

%%%%
\subsection{Results}

At the end of the experiment, the metrics showed that the developers which used CollabIDE obtained 
a higher completion percentage than those who used the other IDEs. The developers who used 
CollabIDE also spent significantly less time doing actions related to version control than the 
developers that used other IDEs.


\begin{figure}[htbp]
  \centering
  \includegraphics[width=0.7\textwidth]{img/resultsTableCollaborative}
  \caption{Distributed Development experiment results}
  \label{fig:collabide}
\end{figure}

\begin{figure}[htbp]
  \centering
  \includegraphics[width=0.7\textwidth]{img/completionCollaborative}
  \caption{Completion percentage graph for Distributed Development}
  \label{fig:collabide}
\end{figure}

\begin{figure}[htbp]
  \centering
  \includegraphics[width=0.7\textwidth]{img/versionControlCollaborative}
  \caption{Time spent in version control graph for Distributed Development}
  \label{fig:collabide}
\end{figure}

\begin{figure}[htbp]
  \centering
  \includegraphics[width=0.7\textwidth]{img/resultsTableProductLine}
  \caption{Product Line Development experiment results}
  \label{fig:collabide}
\end{figure}

\begin{figure}[htbp]
  \centering
  \includegraphics[width=0.7\textwidth]{img/completionProductLine}
  \caption{Completion percentage graph for Product Line Development}
  \label{fig:collabide}
\end{figure}

\begin{figure}[htbp]
  \centering
  \includegraphics[width=0.7\textwidth]{img/versionControlProductLine}
  \caption{Time spent in version control graph for Product Line Development}
  \label{fig:collabide}
\end{figure}

%%%%
\subsection{Threats to Validity}
In this section we discuss some aspects of the experiments that can put at risk the validity of the results discussed previously. The first one is that the subject sample size is small, having only one control group in the first experiment and two in the second. Small sample seizes can easily lead to bias [REF], and, in this case, the bias would be towards CollabIDE performing better. Another aspect that can be considered a threat is the low duration of each experiment. In real life contexts, software projects where version control systems are employed usually take months if not years to complete. Additionally, the time intervals between each version control operation are longer, whereas in the experiment they needed to be performed each fifteen minutes. These lower times could also lead to bias towards one of the IDEs, as the development workflow wasn’t completely accurate to one carried out in a real life context. 


\endinput