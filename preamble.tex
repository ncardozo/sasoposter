
%---- PACKAGES
%\usepackage{todonotes}
\usepackage{amssymb}
\usepackage{hyperref}
\usepackage[plain]{fancyref}
\usepackage{ifdraft}
%\usepackage{subcaption}
\let\labelindent\relax
\usepackage[inline]{enumitem}
\usepackage{xcolor}
\usepackage[final]{graphicx}
\usepackage{xspace}
\usepackage[final]{listings}
\usepackage{acronym}
\usepackage{url}
\usepackage{amsmath}
\usepackage{amssymb}
\usepackage[square,numbers,sectionbib]{natbib}
\bibliographystyle{abbrvnat}

\usepackage{etoolbox}
\makeatletter
\patchcmd{\@makecaption}
  {\scshape}
  {}
  {}
  {}
\patchcmd{\@makecaption}
  {\\}
  {.\ }
  {}
  {}
\makeatother

%\let\refsection\relax
%\usepackage
%  [backend=biber,
%   style=trad-abbrv,
%   maxnames=5,
%   firstinits=true,
%   hyperref=true,
%   natbib=true,
%   url=false,
%   doi=false,
%   defernumbers]{biblatex}
%
%
%----[ Biber ]----
%\addbibresource[datatype=bibtex]{local.bib}
%\addbibresource[datatype=bibtex]{bib/general.bib}
%\addbibresource[datatype=bibtex]{bib/compsci.bib}
%\addbibresource[datatype=bibtex]{bib/learning.bib}
%\nocite{*}
%
%\newbibmacro{name:newformat}{%
%    \printnames{authors}
%%   \textbf{\namepartfamily}  % #1->\namepartfamily, #2->\namepartfamilyi
%%   \textbf{\namepartgiven}   % #3->\namepartgiven,  #4->\namepartgiveni
%%   [prefix: \namepartprefix] % #5->\namepartprefix, #6->\namepartprefixi
%%   [suffix: \namepartsuffix] % #7->\namepartsuffix, #8->\namepartsuffixi
%}
%
%\DeclareNameFormat{newformat}{%
%  \usebibmacro{name:newformat}{\textbf{#1}}{\textbf{#3}}{#5}{#7}%
%  \usebibmacro{name:andothers}%
%%  \nameparts{#1}% split the name data, will not be necessary in future versions
%%  \usebibmacro{name:newformat}%
%%  \usebibmacro{name:andothers}%
%}
%
%\AtEveryBibitem
%{
%   \clearlist{address}
%   \clearfield{date}
%   \clearfield{doi}
%   \clearfield{eprint}
%   \clearfield{isbn}
%   \clearfield{issn}
%   \clearfield{month}
%   \clearfield{note}
%%   \clearfield{pages}
%   \clearlist{location}
%%   \clearfield{series}
%   \clearfield{url}
%   \clearname{editor}
%   \ifentrytype{inproceedings}
%     {\clearfield{day}
%      \clearfield{month}
%      \clearfield{volume}}{}
%}
%
%\DeclareFieldFormat*{title}{\textsl{#1}\isdot}
%\DeclareFieldFormat*{journaltitle}{#1}
%\DeclareFieldFormat*{booktitle}{#1}
%
%\renewbibmacro{in:}{} % supress 'In: ' form
%
%\DeclareSourcemap
% {\maps[datatype=bibtex,overwrite]
%   {% Tag entries (through keywords)
%    \map
%      {\step[fieldsource=booktitle,
%       match=\regexp{[Pp]roceedings}, replace={Proc.}]}
%        \map
%      {\step[fieldsource=booktitle,
%       match=\regexp{[Ii]nternational\s+[Cc]onference}, replace={Intl. Conf.}]}
%    \map
%      {\step[fieldsource=journal,
%       match=\regexp{[Jj]ournal}, replace={Jour.}]}
%    \map
%      {\step[fieldsource=journal,
%       match=\regexp{[Tt]ransactions}, replace={Trans.}]}
%    \map
%      {\step[fieldsource=booktitle,
%       match=\regexp{[Pp]roceedings\s+of\s+the.+[Ee]uropean\s+[Cc]onference\s+in}, replace={European Conf. in}]}
%    \map
%      {\step[fieldsource=booktitle,
%       match=\regexp{In\s+[Pp]roceedings\s+of\s+the.+[Ss]ymposium\s+on}, replace={Symp. on}]}
%    \map
%      {\step[fieldsource=booktitle,
%       match=\regexp{[Pp]roceedings\s+of\s+the.+[Ii]nternational\s+[Cc]onference\s+on}, replace={Intl. Conf. on}]}
%    \map
%      {\step[fieldsource=booktitle,
%       match=\regexp{[Pp]roceedings\s+of\s+the.+[Ii]nternational\s+[Ww]orkshop\s+on}, replace={Intl. Workshop on}]}}}
%

%color
\definecolor{OliveGreen}{rgb}{0,0.6,0.3}

%References
%% Listings
\def\fref{\Fref} % treat all \frefs as \Frefs
\renewcommand{\lstlistingname}{Snippet}
\newcommand*{\fancyreflstlabelprefix}{lst}
\newcommand*{\Freflstname}{\lstlistingname}
\newcommand*{\freflstname}{\MakeLowercase{\lstlistingname}}
\Frefformat{vario}{\fancyreflstlabelprefix}%
  {\Freflstname\fancyrefdefaultspacing#1#3}
\frefformat{vario}{\fancyreflstlabelprefix}%
  {\freflstname\fancyrefdefaultspacing#1#3}
\Frefformat{plain}{\fancyreflstlabelprefix}%
  {\Freflstname\fancyrefdefaultspacing#1}
\frefformat{plain}{\fancyreflstlabelprefix}%
  {\freflstname\fancyrefdefaultspacing#1}

\renewcommand{\tablename}{Table}  
  
% ln delimiter
\newcommand*{\fancyreflnlabelprefix}{ln}
\newcommand*{\Freflnname}{Line}
\newcommand*{\freflnname}{\MakeLowercase{\Freflnname}}
\Frefformat{vario}{\fancyreflnlabelprefix}%
  {\Freflnname\fancyrefdefaultspacing#1#3}
\frefformat{vario}{\fancyreflnlabelprefix}%
  {\freflnname\fancyrefdefaultspacing#1#3}
\Frefformat{plain}{\fancyreflnlabelprefix}%
  {\Freflnname\fancyrefdefaultspacing#1}
\frefformat{plain}{\fancyreflnlabelprefix}%
  {\freflnname\fancyrefdefaultspacing#1}    


%JavaScript definition
\lstdefinelanguage{JavaScript}{
keywords={typeof, new, true, false, catch, function, return, null, catch, switch, var, if, in, for, while, do, else, case, break, throw, this, instanceof},
keywordstyle=\color{purple}\bfseries,
ndkeywords={},
ndkeywordstyle=\color{blue}\bfseries,
identifierstyle=\color{black},
sensitive=false,
comment=[l]{//},
morecomment=[s]{/*}{*/},
commentstyle=\color{OliveGreen}\ttfamily,
stringstyle=\color{OliveGreen}\ttfamily,
morestring=[b]',
morestring=[b]"
}

\lstset{%
  basicstyle=\small\ttfamily,
  aboveskip=0\baselineskip,
  belowskip=0\baselineskip,
  commentstyle=\color{gray}\footnotesize\ttfamily\itshape,
  prebreak= ,
  numberblanklines=false,
  breaklines,
  numberstyle=\tiny\color{gray}, 
  numbersep=0pt,
  escapechar=`}

\lstdefinestyle{floating}{%
  frame=none,
  float=htb,
  captionpos=b,
  aboveskip=0pt,
  belowskip=0pt
}

% context traits listings
\lstdefinestyle{ctxtraits}
 {language=JavaScript,
  frame=lines,
  showstringspaces=false,
  keywordstyle=\tt\bf,
  tabsize=3,
  style=floating,
  morekeywords={Trait, cop, proceed, Context, activate, deactivate, adapt, addObjectPolicy, manager}
}

%context traits environment    
\lstnewenvironment{ctxtraits}[1][]
 {\lstset{style=ctxtraits,#1}}{}  


 % Context Traits in line source-code
\newcommand{\scode}[1]{\textrm{\texttt{#1}}}
\def\scode{\lstinline[style=ctxtraits]}

%----[ Commands ]---
%Latins
\newcommand{\eg}{\emph{e.g.,}\xspace}
\newcommand{\ie}{\emph{i.e.,}\xspace}
\newcommand{\cf}{\emph{cf.}\xspace}

\newcommand{\ctx}[1]{\texttt{\textsc{#1}}}


%comments
% xcolor
\definecolor{author}{rgb}{.5, .5, .5}
\definecolor{comment}{rgb}{.1, .0, .9}
\definecolor{note}{rgb}{.9, .4, .0}
\definecolor{idea}{rgb}{.1, .7, .0}
\definecolor{missing}{rgb}{.9, .1, .0}


\newcommand{\authorcomment}[3][comment]
  {\ifdraft{\noindent
      \fbox{\footnotesize\textcolor{author}{\textsc{#2}}}
      \textcolor{#1}{\textsl{#3}}}{}}

%% Space-squeezing stuff

\let\origSubsubsection\subsubsection
\renewcommand*{\subsubsection}[1]%
  {\vspace{-1em}\origSubsubsection{#1}}

\makeatletter

% Squeeze figures
\let\orig@figure\figure
\renewcommand*{\figure}[1][]{\orig@figure[#1]\vspace{-1em}} % before
\let\orig@endfigure\endfigure
\renewcommand*{\endfigure}{\vspace{-1ex}\orig@endfigure} % after

% Squeeze tables
\let\orig@table\table
\renewcommand*{\table}{\orig@table\vspace{-1em}} % before
\let\orig@endtable\endtable
\renewcommand*{\endtable}{\vspace{-1ex}\orig@endtable} % after

% Squeeze snippets
\let\orig@lstlisting\lstlisting
\renewcommand*{\lstlisting}{\orig@lstlisting\vspace{-2em}} % before
\let\orig@endlstlisting\endlstlisting
\renewcommand*{\endlstlisting}{\vspace{-3ex}\orig@endlstlisting} % after



% Squeeze captions
\usepackage{caption}
\captionsetup[figure]{aboveskip=1ex,belowskip=-1em}
\captionsetup[table]{aboveskip=1ex,belowskip=-1em}

% Squeeze section titles
%\let\orig@section\section
%\renewcommand*{\section}[1]{\orig@section{#1}\vspace{-.2ex}}

% Squeeze subsections
%\let\orig@subsection\subsection
%\renewcommand*{\subsection}[1]{\orig@subsection{#1}\vspace{-.2ex}}

\makeatother


\acrodef{AST}{Abstract Syntax Tree}
\acrodef{COP}{Context-oriented Programming}
\acrodef{MAS}{Multi-Agent System}
\acrodef{RL}{Reinforcement Learning}
\acrodef{ROP}{Role-oriented Programming}
\acrodef{SOC}{Service-oriented Computing}


