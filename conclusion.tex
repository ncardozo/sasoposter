% $  Id: conclusion.tex  $
% !TEX root = main.tex

%%
\section{Conclusion}
\label{sec:conclusion}

CollabIDE is a development environment prototype for distributed teams, incorporating automated versioning and enabling the generation of multiple software products according to the code's versions base.

CollabIDE is developed with the goal of increasing the productivity of software development teams by reducing the time developers must dedicate to tasks other than coding. CollabIDE specifically targets the versioning and product variant deployment tasks of the software development process. CollabIDE is based on \ac{COP} to tackle both tasks. Contextual situations in which the development task take place (\eg user, time, code extract of the modification) are defined as contexts adaptations, which can be use directly to represent and define the versions of the system. Extraction, definition, and usage of such contexts/versions takes place autonomously, so developers need no to be distracted by these activities. 
\ac{COP} enables a flexible composition of software systems while they execute. Using this feature it is possible to compose different versions as part of a fully working program variant. Composition of such variants requires developers to select which of the contexts they want to use. Available contexts in the project are accessible to developers within the tool itself, in order to minimize the disruption and time spent in generating the variants.

The validation evidences that CollabIDE effectively makes a step forward towards reducing the overhead of dealing with version control systems and product variant generation, when working in a distributed software development team.
Although the scale of the experiment is small, it already shows the potential that an approach like 
CollabIDE has. Thanks to the flexibility of the technology used to create CollabIDE.......

As future work, we plan to conduct a more extensive case study to evaluate the relevance of CollabIDE in larger development teams. Additionally, we see the value of the tool for code reviews, therefore we want to extend CollabIDE with code productivity metrics (\eg Lines of Code, code clones) and analysis facilities~\cite{lienhard12}. Such extension would enable code reviewers to better evaluate the productivity of particular developers. CollabIDE, can be particularly useful in academic/teaching environments to quickly differentiate how much a particular student participated in an assignment, when did the majority of the work took place.

\endinput
