% $  Id: conclusion.tex  $
% !TEX root = main.tex

%%
\section{Conclusion}
\label{sec:conclusion}

CollabIDE is a development environment prototype for distributed teams, incorporating automated 
versioning and enabling the generation of multiple software products according to the code's version
base.

CollabIDE is developed with the goal of increasing the productivity of software development teams by 
reducing the time developers must dedicate to tasks other than coding; specifically targeting the 
versioning and product variant deployment tasks of the software development process. CollabIDE is 
based on \ac{COP} to tackle both tasks. Situations in which the development task take place (\eg user, 
time, code extract of the modification) are defined as contexts, which can be use directly to represent 
and define the versions of the system. Extraction, definition, and usage of such contexts/versions takes 
place autonomously, so developers need not to be distracted by these activities. 
\ac{COP} enables flexible composition of software systems while they execute. Using this feature it is 
possible to compose different versions as part of a fully working program variant. Composition of such 
variants requires developers to select which of the contexts they want to use. Available contexts in the 
project are accessible to developers within the tool itself, in order to minimize the disruption and time 
spent in generating the variants.

The validation of CollabIDE evidences its effectiveness in reducing the overhead of dealing with version 
control or product variant generation systems, when working in a distributed software development team 
or when employing the product lines development model.
Although the scale of the experiment is small, it already shows the potential that an approach like 
CollabIDE has. Thanks to the flexibility of the technology used to create CollabIDE, it can easily be 
extended with features that are common in other IDEs like code linting, auto-completion, support for 
other programming languages and debugging among others. One important extension that could be 
made is the possibility of managing various files and folders as a software development project rarely 
relies on only one file. Furthermore, there are many ways the existing features can be tweaked to make 
them better at approaching the problem they intend to solve. Some examples would be introducing 
additional code highlighting, a code history feature that lets users navigate to a specific change made 
or even letting different product variants be merged in a similar fashion of how it’s done on git. Another 
limitation of CollabIDE that is worth discussing is the maximum amount of developers in a project. Currently CollabIDE supports
up to 8 developers. This is due to the need to identify each one with an easily distinguishable color.
If the number of developers was higher it would get increasingly difficult to choose colors that can be distinguished from one another.
This limitation is worth approaching for the case of open-source projects
where there can be a high number of developers lending their contributions in single or multiple files.
As future work, we plan to conduct a more extensive case study to evaluate the relevance of CollabIDE 
in larger development teams. Additionally, we see the value of the tool for code reviews, therefore we 
want to extend CollabIDE with code productivity metrics (\eg Lines of Code, code clones) and analysis 
facilities~\cite{lienhard12}. Such extension would enable code reviewers to better evaluate the 
productivity of particular developers. CollabIDE, can be particularly useful in academic/teaching 
environments to quickly differentiate how much a particular student participated in an assignment, or 
when the majority of work took place.


\endinput
