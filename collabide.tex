% $  Id: collabide.tex  $
% !TEX root = main.tex

%%
\section{CollabIDE}
\label{sec:collab-ide}

CollabIDE is a cloud-based integrated development environment presenting collaboration facilities 
distributed development teams working on a software project. CollabIDE has three key features which 
help to reduce the overhead of managing versions or product variants. These features are:
\begin{enumerate*}[label=(\arabic*)] 
\item version management, 
\item product variant management, and 
\item concurrent development.
\end{enumerate*}

%%
\subsection{Version Management}
\label{sec:vcs}
Software products evolve over time~\cite{lehman02}. Rather than releasing a fully working product at 
once, developers implement products incrementally. Each product increment focuses on a single 
feature or part of the functionality at a time. Each of these features marks a new \emph{version} of the 
system, moving forward linearly in time. Defining and synchronizing between these pieces of 
functionality must be carried out multiple times in a project's lifespan. In the long-term, this process 
has a lasting impact in the productivity.

CollabIDE takes the association between a feature's development, and program versions to an 
extreme. In CollabIDE, every code modification (regardless of its size) is marked as a new product 
increment. Furthermore, instead of asking developers to continuously create and manage product 
versions, CollabIDE automatically generates versions for each code increment. These increments are 
identified by a set of variables relevant to the project development. For example, relevant information 
to generate versions may include the developer's name, or the time of the day. Additionally, if 
developers want to create their own versions, it is posible to do so simply by giving them a name in 
the versions pane of the interface. \fref{fig:versions} shows the way product versions are 
managed in the CollabIDE. 

A differentiating feature of CollabIDE is that versions are live. That is, every time a new version is 
created by any of the developers, this is made available to all other developers immediately. 
As a consequence, switching between versions, developers have an immediate view of the complete 
state of the code at the moment the version was created or edited. 

\begin{figure}[tbp]
  \centering
  \includegraphics[width=0.7\textwidth]{img/fig4-collabIDEVersionManagement}
  \caption{Version visualization and management in CollabIDE}
  \label{fig:versions}
\end{figure}

The facilities for version management in CollabIDE help reduce the long-term impact of creating, 
updating, and synchronizing product versions, as these processes occur autonomously upon simple 
interactions.

%%
\subsection{Product Variant Management}
\label{sec:product-variant}
In \ac{SPL} development product variants are managed through variability models, describing the 
different components of the product, and the different options for each of these 
components~\cite{pohl05}. A product variant consists of the composition of a base product with 
different features, usually specialized for each costumer or purpose. This process requires developers 
to define beforehand both, the base product, and the variations points (\ie those features that can be 
interchanged with other features). This process is time consuming for the project setup, and fixed in 
the points where variations may take place.

CollabIDE takes advantage of its \ac{VCS} to enable a more flexible product variation model. Product 
variants are defined by the composition of the different versions available in the system. These 
versions may be chosen based on the time of their creation (as explained in the previous section), or 
taking into account the contributions made by specific developers. The composition takes place 
behind the scene using \ac{COP} facilities for dynamic software composition (\cf 
\fref{sec:implementation}).

\fref{fig:contribution} shows the panel in the CollabIDE interface to select the different variants, in this 
case according to developers' contributions. Selecting one of the contributions renders the associated 
code as part of the interface, and composes the variant with the currently working code. From a 
product variant point of view, this feature eases the process of building new variants from existing 
fragments of code, effectively reducing the overhead derived from executing initial configuration 
processes.

\begin{figure}[htbp]
  \centering
  \includegraphics[width=0.7\textwidth]{img/fig3-collabIDEContributionManagement}
  \caption{Product composition through contribution selection}
  \label{fig:contribution}
\end{figure}


%%
\subsection{Concurrent Development}
Using CollabIDE, developers are able to work concurrently on the same code base offering immediate feedback about the contributions of other developers.
Concurrent development means that all the actions a developer does in the IDE are synchronized. This lets all developers view in real time what changes are being made by other team members in the code editor or the control panel. In \fref{fig:layers} we show how the IDE highlights the code belonging to a certain team member. This feature eliminates the need of constantly having to obtain the latest changes made by other team members. Additionally, conflicts become nonexistent as there will always be only one version of the content being seen by the developers at a given time. The awareness of what each team member has done increases too with this feature, boosting coordination and allowing developers to know in little time what progress has been made in a project since the last time they worked on it. 

In a project, each developer is assigned a unique color which is used to highlight the contributions of code said developer has done. 

\begin{figure}[htbp]
  \centering
  \includegraphics[width=0.7\textwidth]{img/fig2-collabIDEConcurrentProgramming}
  \caption{First-hand view of developers' contributions}
  \label{fig:layers}
\end{figure}


%%
\subsection{Implementation}
\label{sec:implementation}

The implementation of CollabIDE reconciles \ac{COP} with developing software systems integrated as part of a cloud-based platform.

%%%%
\subsubsection{\ac{COP} in a Nutshell}
The base of CollabIDE is \acf{COP}. \ac{COP} is a programming paradigm designed for the dynamic adaptation of software systems based on context information gathered from the system's surrounding execution environment~\cite{salvaneschi+12survey}. \ac{COP} consists of three main abstractions: \emph{Contexts}, \emph{Behavioral adaptations}, \emph{Adaptation activation}.

\paragraph{Contexts} represent semantically meaningful situations gathered from the systems surrounding environment~\cite{dey01}. Contexts are defined as first-class entities .

\paragraph{Behavioral adaptations,} are fine-grained behavior definitions associated to one or several context entities. Behavioral adaptations dictate the observable behavior of a system under particular situation sensed from the environment. The combination of behavioral adaptations and contexts constitute what we call and \emph{adaptation}.

\paragraph{Adaptations activation,} and their different combinations, are visible in the system through the activation and deactivation of context entities, effectively incorporating and withdrawing the behavioral adaptations associated with such contexts.


%%%%
\subsubsection{CollabIDE Internals}
To implement the features described previously, various contexts must be considered. Each developer 
and each product version represents a different context. To achieve this context management, we used 
the \ac{COP}, specifically using Context Traits~\cite{gonzalez13}. 
Context Traits provides the necessary elements to define a specific behavior during runtime for each 
context. CollabIDE features were build making use of these elements, which helped define the 
contribution state of a developer among other elements.
For the code editor of CollabIDE we used Firepad\footnote{\url{https://firepad.io/\#1}} which is an open 
source collaborative text editor. Firepad stores all the changes made in the editor in Firebase’s real 
time non relational database. Additional to the data Firepad stores, we store user and product versions 
data in Firebase.

\endinput
