% $  Id: collabide.tex  $
% !TEX root = main.tex

%%
\section{CollabIDE}
\label{sec:collab-ide}

CollabIDE is an integrated development environment on the cloud that lets development teams 
collaborate on a software project. CollabIDE has three key features which help reduce the overhead 
problems described previously. These features are concurrent development, contribution management 
and version management.
Concurrent development means that all the actions a developer does in the IDE are synchronic. This 
lets all developers view in real time what changes are being made by other team members in the code 
editor or the control panel. This feature eliminates the need of constantly having to obtain the latest 
changes made by other team members. 
In a project, each developer is assigned an unique color which is used to highlight the contributions of 
code said developer has done. Additionally, each developer has a contribution state that determines if 
his contributions are visible or not to the rest of the team. With contribution management, any 
developer can easily identify the contributions of each team member and toggle their visibility as 
needed. This feature helps mainly in the resolution of conflicts and boosts the awareness of the work 
done by each team member.
There can be many product versions at a time in a project. Developers have the option of easily 
creating new product versions which are snapshots of the current contents of the code editor. Thanks 
to the concurrent development feature, all the team members obtain a new version the moment it is 
created. The developers can also switch between versions and view immediately the contents of the 
editor at the time the version was created or edited. The time required for managing different product 
versions is reduced significantly with this feature.
To implement the features described previously, various contexts must be considered. Each developer 
and each product version represents a different context. To achieve this context management, we used 
the \ac{COP} paradigm~\cite{salvaneschi+12survey}, specifically the Context Traits implementation. 
Context Traits provides the necessary elements to define a specific behavior during runtime for each 
context. CollabIDE features were build making use of these elements, which helped define the 
contribution state of a developer among other elements.
For the code editor of CollabIDE we used Firepad\footnote{\url{https://firepad.io/\#1}} which is an open 
source collaborative text editor. Firepad stores all the changes made in the editor in Firebase’s real 
time non relational database. Additional to the data Firepad stores, we store user and product versions 
data in Firebase.


\endinput