% $  Id: validation.tex  $
% !TEX root = main.tex


%%
\section{Related Work}
\label{sec:related}

Different tools exist for collaborative writing or coding.
In this section we discuss the features available in such tools and put them in perspective to CollabIDE.

\paragraph
{Collaborative writing platforms} like Google Docs and 
meetingwords\footnote{\url{http://meetingwords.com}} are not designed for code editing, 
however, they employ mechanics similar to those of CollabIDE’s. In Google Docs, changes made to a 
document are asynchronous and saved automatically. Contributions by collaborators are visible in real 
time by following user-specific markers. However,  changes are all merged into a single document as 
users edit. Nonetheless, color highlighting is used to highlight changes made by individual collaborators 
when reviewing a document’s history. Meetingwords provides similar features to those in Google Docs, 
with the additional feature of color highlighting for the editions each user contributes to the document. 
Such color highlighting served as inspiration for the same feature in CollabIDE. However, this is just a 
visual aid to see who contributed each part of the text, and cannot be manipulated as versions in 
CollabIDE.

\paragraph
{Distributed tools for code editing} targeting development productivity have also been 
explored.~\citet{ghorashi} present Jimbo, an IDE that allows developers to collaborate on a common 
project. Jimbo's most relevant features are synchronic and asynchronic collaboration, and user 
identification. The first one lets developers make changes without worrying about conflicts. The second 
one provides version management by allowing developers to follow specific fragments of code in a 
project to later receive notifications if said fragment received any modifications. This second feature may 
not be optimal on big projects where there are thousands of lines of code as it requires continuous user 
intervention, effectively creating an overhead that harms productivity. 
CodeBunk\footnote{\url{https://codebunk.com}}. is a cloud-based tool enabling playback. Playback 
saves the history of all changes made in an instance of the code editor. Users can playback these 
changes, showing each one in the order they were made. This feature reduces conflicts and improve 
awareness in a project. Similar to Jimbo, an overhead is introduced as developers need to interrupt 
their workflow and sit through a history’s playback to identify possible conflicts and changes made by 
others. If an history gets long enough, the interruption times also get longer, and more time is needed 
to get back into the coding workflow.

In CollabIDE, both of these overhead problems are avoided by providing passive aids to developers for 
(automatic) code versioning, and identifying distinct user’s contributions (code highlighting). This way 
developers can focus on writing code rather than on performing additional actions.


\endinput

