% $ Id: introduction.tex  $
% !TEX root = main.tex

%%
\section{Introduction}
\label{sec:introduction}

\acp{VCS}, as subversion\footnote{\url{http://subversion.apache.org/}} and 
Git,\footnote{\url{http://git-scm.com}} are now fundamental to software development processes, as they 
improve productivity by reducing the risk of losing progress and enable to parallelize coding efforts for 
development and maintenance activities in a project~\cite{spinellis05}.
The main benefits of using \acp{VCS} in software development processes are:
\begin{enumerate*}[label=(\arabic*)]
\item traceability over each file's change history, 
\item concurrent editing, and 
\item branching and merging of code extracts on independent streams of changes. 
\end{enumerate*}
Asa cause of the great success of \acp{VCS}, the \ac{DSD} development model gained substantial popularity, as it opens the possibility for developers in different locations and time-zones working on a single product with minimum disruption. This model brings various advantages to 
development teams like costs reduction, increased productivity and ease of finding human talent 
across the globe. Mechanisms like committing, pushing and merging help team members solve synchronization 
and coordination problems that arise in \ac{DSD}. However, these mechanisms introduce an overhead 
problem that stems from the interruption in the coding workflow the developers make when they 
perform an action related to version control. Initially this overhead problem may not have much impact 
on the project but in the long term, the productivity of the team can be negatively affected.
Another software development model that has acquired popularity due to the competitive advantages 
it brings to development teams is the \ac{SPL} development model~\cite{pohl+05sple}. Like the \ac{DSD} 
model, the SPL also suffers from an overhead problem. In this case the problem is caused by the initial 
coding and configuration that must be done at the beginning of a project.

This paper presents CollabIDE, a collaborative environment for 
distributed development designed for automated versioning and product line generation. since the goal 
of CollabIDE is to increase developers' productivity, we conducted a small empirical study with users 
working in distributed teams, measuring the time to complete different tasks using versioning, and the 
ease to release different products' functionality based on recorded versions. 

With CollabIDE, we aim to solve the overhead problems that exists in these development models with 
features that aim to reduce the time developers must spend doing actions related to version control or 
setting up a project that uses \acp{SPL}.



\endinput