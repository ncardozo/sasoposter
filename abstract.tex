% $  Id: conclusion.tex  $
% !TEX root = main.tex

\begin{abstract}
Current practices in software development are anchored in the use of versioning control systems to manage the development progress, or drive the release process. Despite the benefits, using such systems requires developers to interrupt their workflow in order to interact with the versioning system, affecting their productivity. 

\end{abstract}


\endinput
versiones, la enorme cantidad de conveniencias que estos sistemas traen al proyecto hacen que se vuelvan indispensables para el equipo. A pesar de los beneficios que estos sistemas traen, su uso hace que los desarrolladores incurran en un costo adicional de productividad, este costo se origina en la necesidad de los desarrolladores de interrumpir su flujo de trabajo de codificación para llevar a cabo operaciones relacionadas al control de versiones que en algunos casos pueden ser demoradas. En este trabajo presentamos a CollabIDE, un ambiente integrado de desarrollo online que facilita el desarrollo colaborativo alrededor de un proyecto de software y cuyas características están diseñadas para reducir el tiempo que un desarrollador debe invertir en realizar operaciones de versionamiento. A través de un experimento demostramos la efectividad de CollabIDE en reducir el overhead de los sistemas de control de versiones en un modelo de desarrollo dsitribuido y uno modelo de desarrollo basado en líneas de producto.  